\documentclass{article}
\usepackage[utf8]{inputenc}

\title{PS7}
\author{Morgan Nash}
\date{March 2018}

\usepackage{natbib}
\usepackage{graphicx}

\begin{document}

\maketitle



\section{Summary Statistics}
\begin{table}[!htbp] \centering 
  \caption{} 
  \label{} 
\begin{tabular}{@{\extracolsep{5pt}}lccccc} 
\\[-1.8ex]\hline 
\hline \\[-1.8ex] 
Statistic & \multicolumn{1}{c}{N} & \multicolumn{1}{c}{Mean} & \multicolumn{1}{c}{St. Dev.} & \multicolumn{1}{c}{Min} & \multicolumn{1}{c}{Max} \\ 
\hline \\[-1.8ex] 
logwage & 1,669 & 1.625 & 0.386 & 0.005 & 2.261 \\ 
hgc & 2,229 & 13.101 & 2.524 & 0 & 18 \\ 
tenure & 2,229 & 5.971 & 5.507 & 0.000 & 25.917 \\ 
age & 2,229 & 39.152 & 3.062 & 34 & 46 \\ 
\hline \\[-1.8ex] 
\end{tabular} 
\end{table}

Whether logwage is missing does not appear to depend on other values; therefore, the data is MCAR. 

\section{Regression Comparison}
\begin{table}[!htbp] \centering 
  \caption{} 
  \label{} 
\begin{tabular}{@{\extracolsep{5pt}}lccc} 
\\[-1.8ex]\hline 
\hline \\[-1.8ex] 
 & \multicolumn{3}{c}{\textit{Dependent variable:}} \\ 
\cline{2-4} 
\\[-1.8ex] & \multicolumn{3}{c}{logwage} \\ 
\\[-1.8ex] & (1) & (2) & (3)\\ 
\hline \\[-1.8ex] 
 hgc & 0.062$^{***}$ & 0.049$^{***}$ & 0.049$^{***}$ \\ 
  & (0.005) & (0.004) & (0.004) \\ 
  & & & \\ 
 collegenot college grad & 0.146$^{***}$ & 0.160$^{***}$ & 0.161$^{***}$ \\ 
  & (0.035) & (0.026) & (0.026) \\ 
  & & & \\ 
 tenure & 0.023$^{***}$ & 0.015$^{***}$ & 0.015$^{***}$ \\ 
  & (0.002) & (0.001) & (0.001) \\ 
  & & & \\ 
 age & $-$0.001 & $-$0.001 & $-$0.001 \\ 
  & (0.003) & (0.002) & (0.002) \\ 
  & & & \\ 
 marriedsingle & $-$0.024 & $-$0.029$^{**}$ & $-$0.029$^{**}$ \\ 
  & (0.018) & (0.014) & (0.014) \\ 
  & & & \\ 
 Constant & 0.639$^{***}$ & 0.833$^{***}$ & 0.834$^{***}$ \\ 
  & (0.146) & (0.115) & (0.115) \\ 
  & & & \\ 
\hline \\[-1.8ex] 
Observations & 1,669 & 2,229 & 2,229 \\ 
R$^{2}$ & 0.195 & 0.132 & 0.131 \\ 
Adjusted R$^{2}$ & 0.192 & 0.130 & 0.129 \\ 
Residual Std. Error & 0.346 (df = 1663) & 0.311 (df = 2223) & 0.311 (df = 2223) \\ 
F Statistic & 80.508$^{***}$ (df = 5; 1663) & 67.496$^{***}$ (df = 5; 2223) & 66.909$^{***}$ (df = 5; 2223) \\ 
\hline 
\hline \\[-1.8ex] 
\textit{Note:}  & \multicolumn{3}{r}{$^{*}$p$<$0.1; $^{**}$p$<$0.05; $^{***}$p$<$0.01} \\ 
\end{tabular} 
\end{table}
The highest hgc coefficient is found using the complete cases regression. This makes some sense because using complete cases as opposed to other imputation methods does not reduce the variation in the dependent variable for which the explanatory variables can account. The two imputation methods produced the same coefficient and a similar R-squared that is lower than that of the complete cases. The other coefficients in these two imputation methods are similar, as well. It seems that complete cases comes closest to the true value of Beta1.

\section{Project Progress}
I am going to continue to refine my analysis of Government Restrictions on Religion. I will look closer into economic variables and the GRI, and I will also add the Social Hostilities Index from Pew as an explanatory variable. I want to try different cleaning methods on the data from what I did when I first conducted my analysis, so I will be pulling data from Pew, the UN, and the World Bank and trying out the imputation methods from this homework.

\end{document}
